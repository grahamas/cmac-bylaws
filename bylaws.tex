% * Preamble
\documentclass{article}
% ** Packages
\usepackage[latin9]{inputenc}
\usepackage[unicode=true,pdfusetitle,
 bookmarks=true,bookmarksnumbered=false,bookmarksopen=false,
 breaklinks=false,pdfborder={0 0 1},backref=section,colorlinks=false]
 {hyperref}
\hypersetup{
 colorlinks,citecolor=black,filecolor=black,linkcolor=red,urlcolor=black,linktocpage}

\makeatletter

%%%%%%%%%%%%%%%%%%%%%%%%%%%%%% LyX specific LaTeX commands.
%% Because html converters don't know tabularnewline
\providecommand{\tabularnewline}{\\}

%%%%%%%%%%%%%%%%%%%%%%%%%%%%%% User specified LaTeX commands.

\usepackage{fullpage}
% ** Title and date
\title{CMAC Bylaws}
\date{April 17, 2018}



\makeatother


\begin{document}
\maketitle

% * Changelog
\noindent \textbf{Changelog:} \\
 %
\begin{tabular}{l}
Amended and updated May 25, 2014 \tabularnewline
Ratified May 23, 2013. \tabularnewline
Amended and updated May 18, 2013. \tabularnewline
Amended April 18, 2013. \tabularnewline
Ratified May 15, 2012. \tabularnewline
Amended and updated May 5, 2012. \tabularnewline
Ratified May 19, 2011. \tabularnewline
Amended and updated April 15, 2011. \tabularnewline
Ratified April 27, 2010. \tabularnewline
Amended and updated April 25, 2010. \tabularnewline
Ratified June 1, 2009. \tabularnewline
Amended and updated April 14, 2009. \tabularnewline
Ratified November 2007. \tabularnewline
Amended and updated October 9, 2007. \tabularnewline
Ratified May 10, 2007. \tabularnewline
\end{tabular}

% * Table of contents
\noindent \newpage{}

\tableofcontents{}

% * Name
\newpage{}
\section{Name}
\begin{enumerate}
\item The name of the group shall be CMAC: The University of Chicago Glee Club.
\item The name shall be changed by motion and second from the floor at the
Annual Meeting, followed by a three-fourths vote in the affirmative
and majority approval of the Executive Board (hereafter ``the Board,'' as
defined in Section \ref{board}).
\end{enumerate}

% * Mission
\section{Mission}

CMAC exists first and foremost to foster the growth and development
of its members both musically and personally. CMAC also strives to
maintain the highest musical quality as an ensemble and to enrich
both the Hyde Park and University of Chicago communities.

% * Membership
\section{Membership} \label{membership}

% ** Definition
\subsection{Definition}

Members:
\begin{enumerate}
  \item have been recommended by the Conductor for Membership; and
  \item have been approved and invited by the Board; and
  \item have accepted the invitation of Membership.
\end{enumerate}
Members who are registered students of the University of Chicago have voting
privileges for meetings of the Membership and Annual Meetings, except where
otherwise specified.  In abiding by Office of the Reynolds Club and Student
Activities rules, Members who are not registered students of the University of
Chicago will have no voting privileges.

The Board shall decide whether or not to invite a Recommendee who is a student
of the University of Chicago by majority vote. In the case where the Recommendee
is not a student of the University of Chicago, the decision to invite shall only
be made by unanimous vote of the Board.

% ** Section
\subsection{Section}

The Conductor shall assign each Member to one of four sections which
correspond to the four musical voice parts for men's voices and to
any sub-sections or additional sections as necessary.

% ** Period of Membership and Dues
\subsection{Period of Membership and Dues}

Excepting resignation or removal from Membership pursuant hereto,
each Member shall retain their Membership for so long as the Member shall remain
a student at the University of Chicago (or, in the case of a non-student,
for one year following their entrance into Membership) and continue
to pay such dues as may be determined under these Bylaws.

% ** Alumni
After three consecutive quarters of membership (or two, if the member's first
quarter was Winter quarter), the member is eligible to become a CMAC Alum
(hereafter simply Alum). Everyone eligible to become a CMAC Alum will be invited
to join the CMAC Alumni Association upon graduation, or upon surrendering their
membership if not a student.

% ** Member Attendance
\subsection{Member Attendance} \label{attendance}
\begin{enumerate}
\item Members shall make every attempt to attend all regular rehearsals,
  meetings, and performances (as defined in Section \ref{gig}).
\item Members who agree to perform in a gig shall
  make all attempts to attend it.
\item Members shall make every attempt to attend special rehearsals, sectionals,
  and meetings (as defined in Section \ref{meeting})
\item A Member who does not attend any rehearsal or performance shall be
  considered absent from that rehearsal or performance.
\item Arrival of a Member to a regular rehearsal or performance after the
  scheduled start time shall be considered a tardy. Two tardies shall equate to
  an absence.
\item Absence of a Member from a regular rehearsal or performance for twenty
  minutes or more shall be considered absent.
\item Prior to an absence or tardy, Members must notify the entire executive
  board of their upcoming absence/tardy via email, providing their reason.
\item Unless excused in advance by the Board, attendance at all Dress Rehearsals
  is mandatory for performance in a concert.
\item A Member who has accrued three absences during one regular academic
  quarter shall be required to have a meeting with the Officers within one week
  of the third absence for review. Officers shall review the Member's
  performance, attendance and prior notifications of absences and decide what
  action to take by a majority of the Board. Decisions shall be made attempting
  in all cases to preserve both the musical and ethical integrity of the group.
\item A member who has accrued three or more absences in one academic quarter
  shall be considered to have excessive absence and faces disciplinary action,
  potentially exclusion from an upcoming performance.
\item Under extenuating circumstances, the above clauses may be lifted on
  an individual basis by a majority vote of the Officers.
\end{enumerate}

% ** Probation
\subsection{Probation}

The Executive Officers of the group may vote by majority whether to
put a Member on probation if:
\begin{enumerate}
\item Said Member is absent three or more times within a quarter; or
\item The Conductor recommends probation for said member; or
\item Said Member fails to attend performance.
\end{enumerate}
The limit and terms of said probation period shall be determined by
the Board and the Conductor but may include the prohibition
of said member from rehearsals, gigs, and/or performances. If the
member in question violates the terms of the probation period the
Board may then begin proceedings to remove said members from the group.

% ** Lapse of Membership
\subsection{Lapse of Membership}
\begin{enumerate}
\item A Membership lapses when a Member takes an extended break from
  participation in CMAC. This Member may be said to be ``on break.''
\item Conditions for lapsing are:
  \begin{enumerate}
  \item The Member communicates to exec in writing that they wish to take a
    break, or that they plan to miss 6 consecutive rehearsals.
  \item Any Member who is absent from 6 consecutive rehearsals is automatically
    considered to be on a break.
  \end{enumerate}
\item Members on break forfeit all voting privileges during their break.
\item If the Member does not return to full participation (as determined by the
  Board) in the quarter following their first full quarter on break, then the
  Member automatically forfeits their Membership, but without forfeiting their
  ability to become an Alum. The Board may grant only a single quarter extension
  on a case-by-case basis.
\item The only exception to the preceding rule is in the case that a Member is
  gone from the University for a maximum of three academic quarters
  (e.g. studying abroad). In this case, the lapse may last for that single
  academic year, with no extension.
\item A Member may return from break to full membership in the first week of
  Fall or Winter quarter, without needing Board approval. A Member may only
  return from break otherwise with a unanimous vote of the full Board, and the
  Board having consulted with the Conductor. In the case where the Board forbids
  a Spring quarter return, the Board may grant a single quarter extension of
  lapse even when it would be otherwise disallowed by these rules.
\end{enumerate}

% ** Removal from Membership
\subsection{Removal from Membership}
\begin{enumerate}
\item Membership may be rescinded under the following conditions:

\begin{enumerate}
\item Uncontrollable tardiness or absenteeism; or
\item Egregiously inappropriate behavior, whether within CMAC or within the
  larger community; or
\item Violation of terms of probation; or
\item Any other behavior which the Board finds to be sufficiently serious and
  detrimental to the group.
\end{enumerate}
\item The following procedure shall be followed regarding the removal of
members from CMAC:

\begin{enumerate}
\item At a Meeting of the Board, one Officer shall first state the reasons
for removal and then shall make a motion to recommend to the Members
that said member (the Accused) shall be removed. A second for the
motion is required.
\item The Board shall then vote on the motion for removal from membership.
A simple majority is required for passage of the motion.
\item If the motion passes, then the Board shall:

\begin{enumerate}
\item immediately inform the Accused in writing of the decision and of
the reasons for the said decision; and
\item at the next regularly scheduled rehearsal, call an emergency meeting
of the Members to discuss and to vote on the removal of the Accused
from membership.
\end{enumerate}
\item The following shall occur at the Emergency Meeting:

\begin{enumerate}
\item The Board shall give their rationale for the recommendation to
remove the Accused from membership.
\item The Accused or the Accused's representative shall then give a defense
of the Accused and shall respond to the rationale given by the Officers.
\item Speaking times for the Board and for the Accused or their representative
shall be equal.
\item The President shall then open the floor to all members (including the
  Accused) who wish to discuss the recommendation to remove the Accused. Every
  member who wishes to speak shall be given the opportunity to do so. The
  President shall announce procedure for the following discussion, including at
  a minimum that none shall speak out of turn and that none shall speak but when
  the President gives them the floor. These rules shall be strictly enforced,
  and multiple violations or egregious violation shall result in expulsion from
  the discussion space and forfeiture of speaking rights within the discussion.
\item When no one else wishes to speak, the Accused shall leave the room
and a Member shall then move to remove the Accused from membership.
A second is required.
\item The Members shall then vote to remove the Accused from membership.
The motion carries if two thirds of those assembled support the motion.
\end{enumerate}
\end{enumerate}
\item Any member removed by the proceedure described in this section is no
  longer eligible to be an Alum.
\end{enumerate}

% * Rehearsals and Performance
\section{Rehearsals and Performance}

% ** Regular Rehearsals
\subsection{Regular Rehearsals}

During each regular academic quarter, Regular Rehearsals shall be held twice
weekly at such times and locations as to be designated by the Board and
Conductor. The day and time of Regular Rehearsals shall be announced to the
Members no later than the Sunday preceding the beginning of the regular academic
quarter. Regular Rehearsals shall be held during all weeks of the quarter except
for Finals Week.

% ** Special Rehearsals
\subsection{Special Rehearsals}

Special Rehearsals are all scheduled rehearsals other than those at the Regular
Rehearsal day and time. Special Rehearsals shall be held at such times and
locations as may be designated by the Board, Section Leaders (\ref{section_leaders}),
and the Conductor. The Board, Section Leaders, and the Conductor shall make
every effort to announce the Special Rehearsal to membership as far in advance
as possible.

% *** Sectional
A Sectional is a Special Rehearsal held for a single Section. Each Section shall
have two Sectionals per quarter, one to be held before the beginning of 5\textsuperscript{th}
week, and one to be held before the beginning of 9\textsuperscript{th} week.

% *** Retreat
A Retreat is an overnight trip including a long Special Rehearsal. The Board
shall make every effort to schedule a Retreat during the first four weeks of
Fall Quarter.

% ** Extension of Rehearsals
\subsection{Extension of Rehearsals}

The Conductor may extend the duration of any Rehearsal by not more
than fifteen minutes without prior notice, but any Member, with notification
and consent of the President, shall be able to leave at the regular
rehearsal time.

% ** Performances
\subsection{Performances}

All performances scheduled for the current academic quarter shall
be announced to the membership at the first regular meeting held after
they are scheduled during said quarter.

% *** End-of-Quarter Concert
The Group shall perform a concert on the weekend following tenth week of each
regular academic quarter. This concert is referred to as the End-of-Quarter Concert.

% *** Gigs
\subsection{Gigs} \label{gig}

A gig shall be defined as any CMAC performance that is not the End-of-Quarter
Concert. A gig shall be scheduled as follows: upon motion of the Board, the
President shall call for volunteers from the Members for a proposed gig. If the
Conductor determines that the number and voice parts of any volunteers are
appropriate for the gig, the same shall be scheduled.

% * Meetings of Membership
\section{Meetings of Membership} \label{meeting}

% ** Annual Meeting
\subsection{Annual Meeting} \label{annual_meeting}

The Board shall hold one meeting of all members before the sixth week
of Spring Quarter in order to deal with group business as needed and in order to
elect a new Board. The meeting shall be announced to the Membership at large no
less than two regular rehearsals immediately preceding the Annual
Meeting.

% ** Special Meetings
\subsection{Special Meetings}

Special Meetings of the members may be called by the President, the Board, or by
petition signed by one-third of the Membership. In the first case, the President
chairs the meeting. In the second case, the Board elects the Chair by majority
vote. In the third case, the petition must specify a Chair. Except in cases of
emergency, the Special Meetings shall be announced no less than two consecutive
regular rehearsals immediately preceding the Meeting. These meetings shall be
called to order for business that requires the considerations of the members.

% ** Quorum
\subsection{Quorum}

Unless otherwise provided by these Bylaws, at any meeting of the Members,
a majority of the Members shall constitute a quorum. At any election
meeting, including the Annual Meeting, two-thirds of the Members shall
constitute a quorum.

% ** Meeting Chair
\subsection{Meeting Chair}

All business meetings of the Members shall be chaired by the President,
or as designated by these Bylaws. The Chair may appoint a member to
record votes, or as otherwise needed to assist them.

% * Executive Board
\section{Executive Board} \label{board}
% ** Executive Board
\subsection{Executive Board}

The Executive Board (hereafter ``the Board'') shall be composed of the Executive
Officers (hereafter ``Officers'') of President, Treasurer, Publicity Chair,
Development Chair, and Secretary. The Board shall be elected at the Annual
Meeting and shall assume office as designated by these Bylaws. The Board shall
be responsible for the management of all CMAC activities, funds, and day-to-day
operations, and shall have all agency of CMAC necessary to carry out these
responsibilities. Each Officer shall be granted one vote on the Board.

All Officer-candidates shall:
\begin{enumerate}
\item Be students of the University of Chicago in good standing;
and
\item Have been Active Members for two of the previous three quarters, including
the current quarter, or in special circumstances, have been approved
by the Membership for candidacy; and
\item If elected, serve for no more than two consecutive years in the same
position; and
\item Plan to be in Chicago for at least two quarters of the academic year
in which they serve; and
\item Inform the Membership during elections of any prolonged absences.
\end{enumerate}

% ** Duties of the Board
\subsection{Duties of the Board}

The Board is responsible for the general administration of the group
and allotment of any and all CMAC monies. No later than the third
rehearsal of every quarter the Board shall publish the requirements
and responsibilities of every member. This document shall include,
but is not limited to, concert dates and times, rehearsal dates and
times, gig dates and times, and any other membership obligations.
The Board shall have sole discretion over invitations to Membership
and oversee any disciplinary action. The Board shall fix dates and
times for all concerts during the academic year as soon as possible.
Upon election of incoming Officers, the outgoing officers are responsible
for training of incoming Officers. Additionally, at the end of their
tenure each officer must prepare a written report of their actions
and thoughts during their tenure.

% ** Description of Offices
\subsection{Description of Offices}

% *** Delegation of Responsibility
\subsubsection{Delegation of Responsibility}

An Officer is personally responsible for completing the duties assigned
to them in these Bylaws.

% *** President
\subsubsection{President}

The President is the chairman of the Board; as such they set agendas for
meetings. Additionally, they shall oversee the execution of all executive
duties. The President is ambassador to both the University and the
Community. They are responsible for storage and maintenance of all CMAC
property, except where otherwise noted. In addition the President is responsible
for keeping CMAC in compliance with all applicable laws and regulations.

% *** Treasurer
\subsubsection{Treasurer}

The Treasurer shall be responsible for supervision of all financial accounts;
the collection, banking, and dispersion of all CMAC monies as deemed appropriate
by the Board. The Treasurer is the signator of all CMAC accounts. The Treasurer
may not, under any circumstance, approve any transaction without the Board's
consent. These responsibilities shall include, but be not limited to, the
planning and implementation of an annual budget, accurate bookkeeping of all
CMAC accounts, and applying for funds. In the Treasurer's quarterly report to
membership and alumni, they should give a financial synopsis including
transactions and current balances.

% *** Publicity Chair
\subsubsection{Publicity Chair}

The Publicity Chair is responsible for recruitment activities and shall also be
responsible for special invitations, concert posters, and concert
advertisement. The Publicity Chair is responsible for all CMAC advertising and
the storage and maintenance of all CMAC advertising materials.

% *** Development Chair
\subsubsection{Development Chair}

The Development Chair shall pursue opportunities related to the internal
and external development of the group including, but not limited to:
soliciting donations, seeking sponsorships, coordinating Alumni donations, and
creating merchandising opportunities. As such, the Development Chair is
responsible for storage and maintenance of CDs. The Development
Chair shall be responsible for the distribution of newsletters to
parents, alumni, and donors.

% *** Secretary
\subsubsection{Secretary}

The Secretary must record Board meeting minutes, and communicate them to CMAC at
large. Additionally, the Secretary is responsible for maintaining accurate
attendance records, and reporting to the Board any truancy worthy of
disciplinary action. The Secretary is responsible for the maintenance of all
CMAC contacts, including but not limited to university administrators, alumni,
and gig contacts.

% ** Meeting of the Officers
\subsection{Meeting of the Board}

The Board shall meet immediately following one rehearsal of every week. At this
meeting each officer should give their weekly report.  Furthermore, the Board
shall listen to any issues brought forth by a Member at this time. Meetings are
open to all Members, but may be closed for sensitive discussions by majority
vote of the Board. Additional meetings may be called to address immediate or
emergency issues. Minutes will be recorded and disseminated by the
Secretary. Any member of the group may attend this meeting.

% ** Election of Officers
\subsection{Election of Officers} \label{elections}

% *** Election Meeting
\subsubsection{Election Meeting}

At the Annual Meeting of the members, as designated by Section
\ref{annual_meeting} of these Bylaws, the Members shall hold elections for the
offices of President, Treasurer, Publicity Chair, Development Chair, and
Secretary. The meeting shall be chaired by the current President or as
designated by these Bylaws, and a non-candidate member shall be appointed by the
Board for the recording of votes. A quorum shall be two-thirds of the
current membership.

% *** Nomination of Candidates
\subsubsection{Nomination of Candidates}

Candidates for office shall be nominated and seconded by any two Members.  Upon
acceptance of nomination, all candidates shall be given the opportunity to
address the membership for a period of time determined by the Chair, not less
than 60 seconds for President-candidates, and not less than 45 seconds for all
other candidates. Members wishing to serve who have not been nominated may
request nomination from the Chair, in which case the Chair shall open the floor
for nominating the Member who has expressed interest in running.

% *** Discussion of Qualifications
\subsubsection{Discussion of Qualifications}

Following address by all candidates, the candidates shall leave the meeting and
non-candidate members shall be given an opportunity to discuss qualifications of
candidates for a period of time decided by a majority of the electoral body
prior to balloting for each office. Time of discussion may be extended as
needed, as shall be decided by a majority vote made from a motion on the
floor. The Chair may also choose to extend the time at their discretion if they
deem it necessary. This discussion is sealed.

% *** Election by Majority
\subsubsection{Election by Majority}

Executive Officers shall be elected by a majority vote of a quorum of the
membership, less the candidates. Meaning, the vote shall still be valid if the
number of members present is less than quorum after the candidates have left the
room, and the majority shall be with respect to the members actually in the
room, not including the candidates. Each member who is present (excluding the
candidates) shall have one vote. If a majority vote of a quorum is not reached
on the first vote, the candidate receiving the lowest number of votes will be
removed from the ballot and the process shall be repeated until a majority is
reached, though the Chair is encouraged to revise the discussion time limit. No
candidate shall return to the room until a majority is reached, and the order of
elimination is sealed and will not be revealed. All votes shall be done by
secret ballot and shall continue until each office is filled.

% *** Order of Elections
\subsubsection{Order of Elections}

The order of elections shall be President, Treasurer, Development
Chair, Publicity Chair, and Secretary. Provided their eligibility and nomination,
a non-elected member may run for more than one office, but if they are
elected to one, they shall immediately drop all candidacy for other
offices. No member shall serve in more than one position on the Executive
Board.
% *** Officers-Elect
\subsubsection{Officers-Elect}
Following election, all Officers-Elect are expected to attend all Board meetings
for the remainder of the quarter. The Officers-Elect hold no vote in these
meetings, but are expected to learn from the incumbent Board. Officers-elect
become Officers with voting privileges and responsibilities after the end of
Spring quarter.

% *** Special Election
\subsubsection{Special Election} \label{special_election}
If an election must be held for a Board position outside of the Annual Meeting,
as in Section \ref{long_absence}, then the Board shall hold a Special
Election. This election will be held by whatever method is deemed expedient by
the Board, such that the election may be carried out in a timely manner while
preserving the secret ballot and while counting the votes of at least a quorum
of the Membership. In the case of an election occurring outside
regular academic quarters, this may not be an in-person election. The election
procedure shall be published one week in advance of the proposed election date,
and shall be carried out unless one-third of the Membership objects in
writing. In such a case, an election method must be proposed and approved by a
majority of the Membership within two weeks. This would be a disaster, so the
Board is strongly encouraged to involve the Membership in the design of their
election method. Should the Membership fail to select a method within two weeks,
then the Board's proposal shall be carried out.

% ** Impeachment of Officers
\subsection{Removal of Officers}

In the case of flagrant incompetence, the Members of CMAC shall, by a
three-fourths vote (two-thirds of Members being a quorum) choose to impeach any
Officer of the Board. The accused officer shall be given no less than two
minutes to make a case for their retainer, which shall immediately be followed by
a vote. A three-fourths vote of the quorum (not including the accused officer)
shall be required for removal. For simple incompetence, the former Officer shall
face no further punishment. If the incompetence extended to behavior
unacceptable for any Member, then further remedies may be proposed, as outlined
in previous sections.

% ** Succession
\subsection{Succession}

% *** Absences Lasting One Quarter or Less
\subsubsection{Absences Lasting One Quarter or Less}
If an Officer will be absent from their Position for one quarter or less, then
the Board (including the to-be-absent Officer, if possible) will elect an Acting
Officer, who shall have all the privileges and responsibilities of the
to-be-absent Officer for the duration of said absence. Until such an election
may be carried out, the Treasurer shall serve as Acting President. An Officer
may opt to hold two positions in such a situation, or may opt to consider their
own Position vacant for the duration of the proposed absence, thus requiring the
subsequent application of this Section to their own Position. However, no
Officer shall have more than one vote. In the case that an Officer opts to hold
two positions (meaning there will only be four votes within the Board), a fifth
vote will be granted to a Member of the group as defined below in Section
\ref{temporary_voting}.

% *** Absences Lasting Longer than One Quarter
\subsubsection{Absences Lasting Longer than One Quarter} \label{long_absence}
If an Officer will be unable to carry out the duties of their position for
longer than one quarter, then that Officer shall forfeit their position on the
Board. In this case, the Board shall hold a Special Election (as defined in
Section \ref{special_election}).

% *** Temporary Executive Voting Privileges
\subsubsection{Temporary Executive Voting Priviledges} \label{temporary_voting}

If the Board should find itself with four or fewer Officers with Executive
Voting rights, the Membership may grant Executive Voting privileges to one of
the Members by majority vote. The Board shall be responsible for nominating such
a Member to the Membership during a regular rehearsal. No other Executive
privileges or duties will be granted to or required of this Member. The Member
shall hold temporary Executive Voting privileges until there exist a full Board
with Executive Voting privileges.

% * Executive Cabinet
\section{Executive Cabinet}

The Board may appoint Executive Cabinet (hereafter ``Cabinet'') members at its
discretion to administer the operational duties of the Board. The Cabinet must
include Ticket Czar, Music Chair, Webmaster, Librarian, and Design
Chair. Additional Cabinet positions may be appointed by the Board as they deem
fit. Cabinet members shall have no vote on the Board.

All Cabinet Appointees shall:
\begin{enumerate}
\item be in Chicago for at least two quarters of the academic year for which
they are appointed; and
\item inform the Board of any conflicts and prolonged absences
during the academic year for which they are appointed.
\end{enumerate}

% ** The Duties of the Executive Cabinet
\subsection{The Duties of the Executive Cabinet}

All Cabinet members shall complete all tasks assigned to their position by the
Board and enumerated herein. Cabinet members will be summoned to
attend at least one Board meeting per quarter, unless instructed
otherwise by the Board, and are encouraged to attend all
Board meetings.

% ** Description of Positions
\subsection{Description of Positions}


\subsubsection{Ticket Czar}

The Ticket Czar shall be responsible for printing, distributing, and tracking
tickets for quarterly performances. The Ticket Czar shall also be responsible
for collecting ticket money from all the members. They shall report to the
Publicity Chair. They shall present a record of ticket sales to the Treasurer.


\subsubsection{Music Chair}

The Music Chair shall collaborate closely with the Conductor in maintaining
and developing the musical integrity of CMAC. This includes, but is
not limited to, aiding the President in running rehearsal, conducting
the group in the absence of the Conductor, and coordinating Sectionals
with Sectional Leaders. In all musical matters they shall defer to the
judgment of the Conductor. They shall report to the Conductor.


\subsubsection{Webmaster}

The Webmaster shall be responsible for all oversight regarding CMAC's
domain, www.cmacsings.com, and other web presence. The Webmaster shall report to the
Publicity Chair.


\subsubsection{Librarian}

The Librarian shall be responsible for maintaining the CMAC catalogue
of music. This includes, but is not limited to, copying and handing
out all new music, and maintaining the records of all old music. The Librarian
shall report to the Conductor.


\subsubsection{Design Chair}

The Design Chair shall be responsible for designing advertising materials,
including but not limited to posters, invitations, and tickets. The Design Chair
shall report to the Publicity Chair.

% ** Appointment of the Executive Cabinet
\subsection{Appointment of Executive Cabinet}

The Board shall appoint all required Cabinet Members following the
Annual Meeting of CMAC. The Board shall announce cabinet positions no
later than two weeks prior to when appointments are made. Members who are
interested in an Executive Cabinet position are encouraged to make their desire
known to the Board a week before appointments are made.

% *** Appointment of Music Chair
\subsubsection{Appointment of Music Chair}

The Board shall consult with and defer to the Conductor when
appointing the Music Chair.

% *** Subsumed Positions
\subsubsection{Subsumed Positions}

In the event of absence of qualified or interested members for a given
Cabinet Position, the duties of the position shall be subsumed by
the Board.

% * Section Leaders
\section{Section Leaders} \label{section_leaders}

% ** Election of Section Leaders
\subsection{Election of Section Leaders}

The Members shall hold elections for section leaders by the fifth
regularly scheduled rehearsal of the year. The Section Leaders shall
be elected, respectively, by the Members of each Section for a term
of one year. Section Leaders must have been Active Members for at
least one academic quarter. Officers are eligible for Section Leader,
but the Membership is encouraged to elect non-Officer Members.

% ** Emergency Election
\subsection{Emergency Election}

In the case of a vacancy occurring more than one month before the
expiration of the term of a Section Leader, the Members of that Section
shall fill such vacancy by special election held as soon as possible
after the vacancy occurs; any Section Leader so elected shall qualify
immediately upon election and shall hold office for the remainder
of the unexpired term.

% ** Duties of Section Leaders
\subsection{Duties of Section Leaders}

Section Leaders shall be the musical leaders of their respective Sections,
and shall:
\begin{enumerate}
\item be the musical and social leaders of their respective section;
\item be responsible for assisting the Conductor in achieving the musical
excellence of the group;
\item maintain clear and accurate records of music rehearsed and musical
notations given during Rehearsals;
\item communicate the same to absent Members of their Sections and otherwise
assist them musically;
\item assist the Secretary in the maintenance of attendance records, the
taking of active steps to discourage absences, and the dissemination
of information to the Members;
\item have the authority to call meetings of their respective sections for
extra practice and to chair or to appoint a chair for said meetings;
\item be responsible for the maintenance and development of traditions for
their section;
\item coordinate with the Social Chair to plan and execute social events
for their section;
\item coordinate with other Section Leaders and with the Social Chair to
plan and execute competitive activities between sections; and
\item shall attend the first Board meeting after their election.
\end{enumerate}

% ** Presidential Chair of Section Meetings
\subsection{Presidential Chair of Section Meetings}

At the President's discretion, the President shall chair (or appoint a chairman
for) any meeting of Members of a Section that the Section Leader cannot
or should not chair.

% ** Succession of Section Leaders
\subsection{Succession of Section Leaders}

If a Section Leader is for any reason unable to serve, there shall
be an immediate vote held by the section to nominate a successor and
replace the Section Leader.

% * Conductor
\section{Conductor} \label{conductor}

CMAC shall have a Conductor responsible for running all rehearsals and
performances. The Conductor shall be recommended by an association of Alumni for
appointment by the Board. The Board shall defer to the Alumni in the complex,
time-intensive process of choosing a Conductor.  The Conductor shall have a
contract created or renewed each year by the Board, regarding expectations and
possible compensation. The Conductor shall not be a student of the University
and hence shall have no vote, but shall always have the right to attend their
meetings and express opinions therein.

% * Finances
\section{Finances}

% ** Signator
\subsection{Signator}

The Treasurer shall be designated signator of any CMAC accounts. Any and all
purchases and/or expenditure from these account shall be made only with the
consent of the Board.

% ** Assessment
\subsection{Assessment}

The Treasurer, with the consent of the Board, shall have the authority
to prescribe reasonable assessments of the Membership including
tour expenses and other necessary items. These assessments shall be
collected by the Treasurer or designee. The Board shall be proactive in avoiding
imposing any financial duress on any Member.

% ** Dues Request
\subsection{Dues Request}

The Treasurer shall submit to the Board an amount to request from
the Membership in member dues. Upon approval of the Board, the Treasurer
or designee shall collect these dues. The Board shall be proactive in avoiding
imposing any financial duress on any Member.

% ** Appeal of Assessments
\subsection{Appeal of Assessments}

Any Member may appeal any financial obligations to the Board within two weeks of
the assessment. The Board may decide to delay, reduce, or cancel any financial
obligation for a particular member to avoid imposing financial duress on any
Member.

% ** Revenues
\subsection{Revenues}

The Ticket Czar shall oversee the distribution and sale of concert
tickets and shall transfer all monies raised to the Treasurer. The
Development Chair shall be responsible for the distribution and sale
of recordings or other merchandise. All sales shall be made at a price
determined by the Board.

% ** Fundraising
\subsection{Fundraising}

The Development Chair shall have the authority to raise funds for CMAC through
any and all available means, and must be familiar with the extensive University
regulations on such activities. The Development Chair shall coordinate the
contact of Alumni in order to solicit contributions to the CMAC accounts a
minimum of once per academic year.

% ** Financial Transparency
\subsection{Financial Transparency}

By the fifth rehearsal of each quarter, the treasurer shall be responsible
for presenting to the membership the status of CMAC's treasury, including
but not limited to: concert revenues, current balance, and projected
cashflow.

% * Indemnification
\section{Indemnification}

To the fullest extent now or hereafter permitted by the laws of Illinois,
CMAC shall indemnify any person who is or was an Officer against expenses
and liabilities in connection with any proceeding involving such Officer
by reason of the Officer having been in such a position.

% * Ratification and Amendment
\section{Ratification and Amendment}
\begin{enumerate}
\item These Bylaws shall be ratified by a two-thirds majority of the Membership
  and shall become effective immediately thereafter.
\item All amendments or additions to these Bylaws shall be ratified by a
  two-thirds majority of the Members present, a quorum requiring three-fourths
  of the Active Members.
\end{enumerate}

% * Discrimination
\section{Discrimination}

CMAC considers students on the basis of individual merit and without regard to
race, color, religion, sex, gender, sexual orientation, national or ethnic
origin, age, disability, or other factors irrelevant to participation in
CMAC. However, CMAC may make requirements based on vocal range or quality that
may result in the group's being composed entirely of one sex. The University of
Chicago has other a cappella groups which provide a substantially equal
environment for students with other voice types, including all-female groups.
\end{document}
