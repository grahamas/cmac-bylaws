% * Preamble
\documentclass{article}
% ** Packages
\usepackage[latin9]{inputenc}
\usepackage[unicode=true,pdfusetitle,
 bookmarks=true,bookmarksnumbered=false,bookmarksopen=false,
 breaklinks=false,pdfborder={0 0 1},backref=section,colorlinks=false]
 {hyperref}
\hypersetup{
 colorlinks,citecolor=black,filecolor=black,linkcolor=red,urlcolor=black,linktocpage}

\makeatletter

%%%%%%%%%%%%%%%%%%%%%%%%%%%%%% LyX specific LaTeX commands.
%% Because html converters don't know tabularnewline
\providecommand{\tabularnewline}{\\}

%%%%%%%%%%%%%%%%%%%%%%%%%%%%%% User specified LaTeX commands.

\usepackage{fullpage}
% ** Title and date
\title{Chicago Men's A Cappella Bylaws}
\date{May 27, 2014}



\makeatother


\begin{document}
\maketitle

% * Changelog
\noindent \textbf{Changelog:} \\
 %
\begin{tabular}{l}
Amended and updated May 25, 2014
Ratified May 23, 2013. \tabularnewline
Amended and updated May 18, 2013. \tabularnewline
Amended April 18, 2013. \tabularnewline
Ratified May 15, 2012. \tabularnewline
Amended and updated May 5, 2012. \tabularnewline
Ratified May 19, 2011. \tabularnewline
Amended and updated April 15, 2011. \tabularnewline
Ratified April 27, 2010. \tabularnewline
Amended and updated April 25, 2010. \tabularnewline
Ratified June 1, 2009. \tabularnewline
Amended and updated April 14, 2009. \tabularnewline
Ratified November 2007. \tabularnewline
Amended and updated October 9, 2007. \tabularnewline
Ratified May 10, 2007. \tabularnewline
\end{tabular}

% * Table of contents
\noindent \newpage{}

\tableofcontents{}

% * Name
\newpage{}
\section{Name}
\begin{enumerate}
\item The name of the group shall be Chicago Men's A Cappella (hereafter
referred to as `CMAC').
\item The name shall be changed by motion and second from the floor at the
Annual Meeting, followed by a three-fourths vote in the affirmative
and approval of the Board.
\end{enumerate}

% * Mission
\section{Mission}

CMAC exists first and foremost to foster the growth and development
of its members both musically and personally. CMAC also strives to
maintain the highest musical quality as an ensemble and to enrich
both the Hyde Park and University of Chicago communities.

% * Membership
\section{Membership}

% ** Definition
\subsection{Definition}

Members:
\begin{enumerate}
\item have been recommended by the Conductor for Membership; and
\item have been approved and invited by the Officers; and
\item have accepted the invitation of Membership.
\end{enumerate}
Members who are registered students of the University of Chicago have
voting privileges for meetings of the Membership and Annual Meetings.
In abiding by Office of the Reynolds Club and Student Activities rules,
Members who are not registered students of the University of Chicago
will have no voting privileges.

% ** Section
\subsection{Section}

The Conductor shall assign each Member to one of four sections which
correspond to the four musical voice parts for men's voices and to
any sub-sections or additional sections as necessary.

% ** Period of Membership and Dues
\subsection{Period of Membership and Dues}

Excepting resignation or removal from Membership pursuant hereto,
each Member shall retain his Membership for so long as he shall remain
a student at the University of Chicago (or, in the case of a non-student,
for one year following his entrance into Membership) and continue
to pay such dues as may be determined under these Bylaws.

% ** Leaves of Absence
\subsection{Leaves of Absence}
\begin{enumerate}
\item A Member may take a leave of absence from active participation in
the Rehearsal and Performance schedule of CMAC without surrendering
any rights as a Member. The Member must notify the President and Conductor
in writing prior to noon of the Saturday before Finals Week of the
quarter prior to the leave of absence. Members may only take one,
quarter-long leave of absence per twelve regular academic quarters
of Membership, and will be referred to in these Bylaws as Non-Active
Members during their leave of absence. Exemptions to this clause may
be granted on a case-by-case basis on approval of the Conductor and
the Board.
\item Any Member who takes a leave of absence from the University of Chicago
of no more than one year shall be given the choice of being either
an Active Member or a Non- Active member. If he wishes to be a Non-Active
Member, he shall reassume a position within CMAC only by submitting
a written notification and receiving consent of the President and
Conductor to do so. If he wishes to be an Active Member, he shall
be allowed to do so, provided that he:

\begin{enumerate}
\item renounces all voting privileges; and
\item abides by the Rehearsal and Performance schedule of CMAC; and
\item resigns any Executive Officer positions in CMAC; and
\item pays such dues as may be determined under these Bylaws.
\end{enumerate}

Exceptions to the above rule may be allowed by a majority vote of
the Membership.

\item Any member studying abroad shall be deemed a Non-Active Member for
as long as he shall be abroad. If he shall have been abroad for up
to one academic year, he shall reassume his position upon return without
notification to the President or Conductor. If he shall have been
abroad for over one academic year, he shall submit written notification
to the President and Conductor and shall reassume his position unless
it is motioned by the Conductor and voted upon by a majority of Officers
to request his expulsion from the group.
\end{enumerate}

% ** Member Attendance
\subsection{Member Attendance}
\begin{enumerate}
\item Active Members shall make every attempt to attend all regular rehearsals,
meetings, and performances (as defined in Section 5.1).
\item Active Members who agree to perform in a gig (as defined in Section
5.5) shall make all attempts to attend it.
\item Active Members shall make every attempt to attend special rehearsals,
sectionals, and meetings (as defined in Section 5.2)
\item An Active Member who does not attend any rehearsal or performance
shall be considered absent from that rehearsal or performance.
\item Absence of an Active Member from a regular rehearsal or performance
after the scheduled start time shall be considered a tardy. Two tardies
shall equate to an absence.
\item Absence of an Active Member from a regular rehearsal or performance
for twenty minutes or more shall be considered absent.
\item In the event of an absence or tardy, active members must notify the executive board of their upcoming absence/tardy via email, providing their reason.
\item Unless excused in advance by the Board, attendance at all Dress Rehearsals
is mandatory for performance in a concert.
\item An Active Member who has accrued three absences during one regular
academic quarter shall be required to have a meeting with the Officers
within one week of the third absence for review. Officers shall review
the Member's performance, attendance and prior notifications of absences and decide what action to
take by a majority of the Officers, the Conductor having no voting
privileges.
\item A member who has accrued three or more absences in one academic quarter
shall be considered to have excessive absence and faces disciplinary
action, potentially exclusion from an upcoming performance.
\item Under extenuating circumstances, the above clauses may be lifted on
an individual basis by a majority vote of the Officers.
\end{enumerate}

% ** Probation
\subsection{Probation}

The Executive Officers of the group may vote by majority whether to
put a member on probation if:
\begin{enumerate}
\item Said member is absent three or more times within a quarter; or
\item The group conductor recommends probation for said member; or
\item Said member fails to attend performance with no reasonable excuse;
and
\end{enumerate}
The limit and terms of said probation period shall be determined by
the Executive Officers and Conductor but may include the prohibition
of said member from rehearsals, gigs and / or performances. If the
member in question violates the terms of the probation period the
Executive Officers and Conductor may then begin proceedings to remove
said members from the group.

% ** Removal from Membership
\subsection{Removal from Membership}
\begin{enumerate}
\item Members may be permenantly removed from CMAC under the following conditions:

\begin{enumerate}
\item Uncontrollable tardiness or absenteeism; or
\item Egregiously inappropriate behavior, either within CMAC or within the
larger community; or
\item Any other behavior which the Officers find to be sufficiently serious
to warrant removal from the group.
\end{enumerate}
\item The following procedure shall be followed regarding the removal of
members from CMAC:

\begin{enumerate}
\item At a Meeting of the Officers, one Officer shall first state the reasons
for removal and then shall make a motion to recommend to the Members
that said member (the Accused) shall be removed. A second for the
motion is required.
\item The Officers shall then vote on the motion for removal from membership.
A simple majority is required for passage of the motion.
\item If the motion passes, then the Officers shall:

\begin{enumerate}
\item immediately inform the Accused through email of the decision and of
the reasons for the said decision; and
\item at the next regularly scheduled rehearsal, call an emergency meeting
of the Members to discuss and to vote on the removal of the Accused
from membership.
\end{enumerate}
\item The following shall occur at the Emergency Meeting:

\begin{enumerate}
\item The Officers shall give their rationale for the recommendation to
remove the Accused from membership.
\item The Accused or the Accused's representative shall then give a defense
of the Accused and shall respond to the rationale given by the Officers.
\item Speaking times for the Officers and for the Accused or his representative
shall be equal.
\item The President shall then open the floor to all members who wish to
discuss the recommendation to remove the Accused. Every member who
wishes to speak shall be given the opportunity to do so. The Accused
may be present and participate in this discussion.
\item When no one else wishes to speak, the Accused shall leave the room
and a Member shall then move to remove the Accused from membership.
A second is required.
\item The Members shall then vote to remove the Accused from membership.
The motion carries if two thirds of those assembled support the motion.
\end{enumerate}
\end{enumerate}
\end{enumerate}

% * Petitions
\section{Petitions}
\begin{enumerate}
\item When a member (as defined in Section 3: Membership) wishes to raise
an issue with the Executive Board, he may submit a petition to the
Executive Board in writing at any time. This petition will be put
up for review and considered by the Executive Board at its next regular
meeting. At their discretion, the Officers may call the petitioner
in order to better understand the issue raised. To pass, it must receive
a majority vote by the Officers concerning the issue raised. The Officers
have to right to call before themselves the petitioner so that he
may defend his petition and so that the Officers might question him.
If the petition passes, the Executive Board must take appropriate
action to ensure that the petitioners request is resolved in a timely
manner.
\item To be considered a valid petition:

\begin{enumerate}
\item It must raise an issue and provide a solution to the problem; and
\item The petitioner must be prepared to answer questions concerning the
petition before the Executive Board; and
\item It may be about any issue pertinent to CMAC.
\end{enumerate}
\end{enumerate}

% * Rehearsals and Performance
\section{Rehearsals and Performance}

% ** Regular Rehearsals
\subsection{Regular Rehearsals}

During each regular academic quarter, Regular Rehearsals shall be
held twice weekly at such times and locations as to be designated
by the Officers and Conductor. The day and time of Regular Rehearsals
shall be announced to the Members no later than the first rehearsal
of the regular academic quarter. Regular Rehearsals shall be held
during all weeks of the quarter except for Finals Week.

% ** Special Rehearsals
\subsection{Special Rehearsals}

Special Rehearsals shall be all other scheduled rehearsals other than
those at the Regular Rehearsal day and time. Special Rehearsals shall
be held at such times and locations as may be designated by the Officers,
Section Leaders (Section 10.2), and the Conductor. The Officers, Section Leaders, and the Conductor shall make every effort to announce the Special Rehearsal to membership as far in advance as possible.

% ** Extension of Rehearsals
\subsection{Extension of Rehearsals}

The Conductor may extend the duration of any Rehearsal by not more
than fifteen minutes without prior notice, but any Member, with notification
and consent of the President, shall be able to leave at the regular
rehearsal time.

% ** Performances
\subsection{Performances}

All performances scheduled for the current academic quarter should
be announced to the membership at the first regular meeting held after
they are scheduled during said quarter.

% ** Gigs
\subsection{Gigs}

A gig shall be defined as a minor performance of CMAC and shall be
scheduled as follows: upon motion of the Executive Board, the President
shall call for volunteers from the Members for a proposed gig. If
the Conductor determines that the number and voice parts of any volunteers
are appropriate for the gig, the same shall be scheduled.

% * Meetings of Membership
\section{Meetings of Membership}

% ** Annual Meeting
\subsection{Annual Meeting}

The Executive Board shall hold one meeting of all members before the
sixth week of Spring Quarter in order to deal with group business
as needed and in order to elect new Officers (as defined in Section
8). The meeting shall be announced to the Membership at large no less
than two consecutive regular rehearsals immediately preceding the
Annual Meeting.

% ** Special Meetings
\subsection{Special Meetings}

Special Meetings of the members may be called by the President, the
Executive Board, or upon the specific request of one third of the
Directors (as defined in Section 7). Except in cases of emergency, the Special Meetings shall be announced no less
than two consecutive regular rehearsals immediately preceding the
Meeting. These meetings shall be called to order for business that requires the considerations of the members.

% ** Quorum
\subsection{Quorum}

Unless otherwise provided by these Bylaws, at any meeting of the Members,
a majority of the Members shall constitute a quorum. At any election
meeting, including the Annual Meeting, two-thirds of the Members shall
constitute a quorum.

% ** Meeting Chair
\subsection{Meeting Chair}

All business meetings of the Members shall be chaired by the President,
or as designated by these Bylaws. The Chair may appoint a member to
record votes, or as otherwise needed to assist him.

% * Board of Directors
\section{Board of Directors}
\begin{enumerate}
\item The Board of Directors shall consist of the Executive Board, Select
Alumni, and Section Leaders, and shall be elected at the Annual Meeting,
or as otherwise designated by these Bylaws. Past Conductors in good
standing shall be honorary members of the Board of Directors and shall
have one vote but shall not be counted as necessary for a quorum.
Alumni appointees shall be nominated by a sitting Member of the Board
or the current Conductor, and shall be elected with a simple majority
of the Members of the Board, a quorum defined in subsection 4. They
shall hold their position for two years and shall be eligible for
renomination for as long as they are in good standing with CMAC. The
current Conductor shall be notified if the Board wishes him to attend
their meetings, and in all circumstances he shall have the right to
attend and speak before the Board whenever he wishes, save for meetings
concerning his employment and conduct, but he shall have no vote and
shall not be counted as necessary for a quorum.
\item The Board shall be responsible for recommending to the current Conductor
and Executive Board such measures as they shall judge necessary and
expedient for their consideration, including but not limited to alumni
relations, performances, Conductor business, and Board elections.
For their services, all non-Active Members of CMAC on the Board of
Directors shall have free access to all CMAC concerts and non-private
gigs and shall be informed of any and all performances and events
regarding CMAC.
\item The Board of Directors shall have Regular Meetings, at a time and
day as determined by the Chairman of the Board (see Section 8.2.7
for description of the Chairman and election thereof), no less than
one per year. These meetings shall be announced no later than one
week before they occur, and all members who shall not be able to attend
shall submit in writing to the Chairman why they shall be absent.
\item At any Board meeting, a quorum shall consist of two-thirds of the
members.
\item At its discretion, the Board may call for a vote of the membership
to decide a specific Board decision, at which time a voting quorum
shall consist of three-fourths of the members.
\end{enumerate}

% * Officers
\section{Officers}

% ** Executive Board
\subsection{Executive Board}

The Executive Board shall be composed of the offices of President,
Treasurer, Secretary, Publicity Chair, and Development Chair. The
Executive Board shall be elected at the Annual Meeting and shall assume
office as designated by these Bylaws. The Executive Board shall be
responsible for the management of all CMAC activities, funds, and
day-to-day operations, and shall have all agency of CMAC necessary
to carry out these responsibilities. Each Officer shall have voting
rights within the Membership. Each Executive Officer shall be granted
one vote on the Executive Board. \\
 All Officer-candidates shall:
\begin{enumerate}
\item Be students of the University of Chicago in good standing;
and
\item Have been Active Members for two of the previous three quarters, including
the current quarter, or in special circumstances, have been approved
by the Membership for candidacy; and
\item If elected, serve for no more than two consecutive years in the same
position; and
\item Plan to be in Chicago for at least two quarters of the academic year
in which the serve; and
\item Inform the Membership during elections of any prolonged absences.
\end{enumerate}

% ** Duties of the Board
\subsection{Duties of the Board}

The Board is responsible for the general administration of the group
and allotment of any and all CMAC monies. No later than the third
rehearsal of every quarter the Board shall publish the requirements
and responsibilities of every member. This document shall include,
but is not limited to, concert dates and times, rehearsal dates and
times, gig dates and times, and any other membership obligations.
The Board shall have sole discretion over invitations to Membership
and oversee any disciplinary action. The Board shall fix dates and
times for all concerts during the academic year as soon as possible.
Upon election of incoming officers, the outgoing officers are responsible
for training of incoming officers. Additionally, at the end of their
tenure each officer must prepare a written report of their actions
and thoughts during their tenure. Additionally, the Executive Board
shall prepare a quarterly debriefing of its activities.

% ** Description of Offices
\subsection{Description of Offices}

% *** Delegation of Responsibility
\subsubsection{Delegation of Responsibility}

An officer is personally responsible for completing the duties assigned
to him in these Bylaws.

% *** President
\subsubsection{President}

The President is the chairman of the Executive Board; as such he sets
agendas for meetings. Additionally, he shall oversee the execution
of all executive duties. He is ambassador to both the University and
the Community. He is responsible for storage and maintenance of all
CMAC property. In addition the President is responsible for keeping
CMAC in compliance with all applicable laws and regulations.

% *** Treasurer
\subsubsection{Treasurer}

The Treasurer shall be responsible for supervision of all financial
accounts; the collection, banking, and dispersion of all CMAC monies
as deemed appropriate by the Board. He is the signator of all CMAC
accounts. The Treasurer may not, under any circumstance, approve any
transaction without the Board's consent. These responsibilities shall
include, but be not limited to, the planning and implementation of
an annual budget, accurate bookkeeping of all CMAC accounts, and applying
for funds. In the Treasurer's quarterly report to membership and alumni,
he should give a financial synopsis including transactions and current
balances.

% *** Secretary
\subsubsection{Secretary}

The Secretary must record Board meeting minutes, and communicate them
to CMAC at large. Additionally, he is responsible for maintaining
accurate attendance records, and reporting to the Board any truancy
worthy of disciplinary action. The Secretary is responsible for the
maintenance of all CMAC contacts, including but not limited to university
administrators, alumni, and gig contacts.

% *** Publicity Chair
\subsubsection{Publicity Chair}

The Publicity Chair is responsible for recruitment activities and
shall also be responsible for special invitations, concert posters,
and concert advertisement. The Publicity Chair is responsible for
all CMAC advertising.

% *** Development Chair
\subsubsection{Development Chair}

The Development Chair shall pursue opportunities related to the internal
and external development of the group including, but not limited to:
soliciting donations, seeking sponsorships, creating merchandising
opportunities, and coordinating Alumni donations. The Development
Chair shall be responsible for the distribution of newsletters to
parents, alumni, and donors.

% *** Chairman of the Board of Directors
\subsubsection{Chairman of the Board of Directors}

Members of the Board of Directors shall elect a Chairman of the Board
from amongst their membership. The Chairman of the Board shall be
responsible for calling, and shall chair, all meetings of the Board
of Directors, as explained in Section 7.2. The Chairman shall report
to the President, and the President shall report to the Executive
Board and Membership at large at his discretion of the measures judged
by the Board to be necessary and expedient for CMAC to consider. The
Chairman of the Board shall receive a quarterly report from the President
as to the status of CMAC, written or oral, at the end of Fall and
Winter quarters. He shall receive a report and explanation of future
group goals from the incoming President. He shall also be signatory
on the CMAC Alumni Account.

% ** Meeting of the Officers
\subsection{Meeting of the Officers}

The Officers shall meet immediately following one rehearsal of every
week. At this meeting each officer should give his weekly report.
Furthermore, the Officers shall listen to any petitions brought to
it under Section 4 at this time. Additional meetings may be called
to address immediate or emergency issues. Minutes will be recorded
and disseminated by the Secretary. Any member of the group may attend
this meeting.

% ** Election of Officers
\subsection{Election of Officers}

% *** Election Meeting
\subsubsection{Election Meeting}

At the Annual Meeting of the members, as designated by Section 6.1
of these Bylaws, the Members shall hold elections for the offices
of President, Treasurer, Secretary, Publicity Chair, and Development
Chair. The meeting shall be chaired by the current President or as
designated by these Bylaws, and a non-candidate member shall be appointed
by the Executive Board for the recording of votes. A quorum shall
be two-thirds of the current membership.

% *** Nomination of Candidates
\subsubsection{Nomination of Candidates}

Candidates for office shall be nominated and seconded by any member.
Upon acceptance of nomination, all candidates shall be given the opportunity
to address the membership for a period of time decided by a majority
of the electoral body prior to balloting for each office. Members
wishing to serve who have not been nominated may request nomination
from the Chair, in which case the Chair shall open the floor for nominating
the Member who has expressed interest in running.

% *** Discussion of Qualifications
\subsubsection{Discussion of Qualifications}

Non-candidate members shall be given an opportunity to discuss qualifications
of candidates for a period of time decided by a majority of the electoral
body prior to balloting for each office. Time of discussion may be
extended as needed, as shall be decided by a majority vote made from
a motion on the floor. The Chair may choose to extend the time by
his discretion if he deems it necessary.

% *** Election by Majority
\subsubsection{Election by Majority}

Executive Officers shall be elected by a majority vote of a quorum
of the membership, as designated by these Bylaws, and each member
shall have one vote. If a majority vote of a quorum is not reached
on the first vote, the candidate receiving the lowest number of votes
will be removed from the ballot and the process shall be repeated
until a majority is reached. If a majority vote between two candidates
cannot be reached, the chair may call for another vote or call for
further discussion. All votes shall be done by secret ballot and shall
continue until each office is filled.

% *** Order of Elections
\subsubsection{Order of Elections}

The order of elections shall be President, Treasurer, Secretary, Development
Chair, and Publicity Chair. Provided his eligibility and nomination,
a non-elected member may run for more than one office, but if he is
elected to one, he shall immediately drop all candidacy for other
offices. No member shall serve in more than one position on the Executive
Board.

% ** Impeachment of Officers
\subsection{Impeachment of Officers}

In the case of flagrant incompetence, the Members of CMAC shall, by
a three-fourths vote (two-thirds of Members being a quorum) choose
to impeach any officer of the Executive Board. The accused officer
shall be given no less than two minutes to make a case for his retainer,
which shall immediately be followed by a vote, in which case a unanimous
vote of the Executive Board (not including the accused officer) and
a three-fourths vote of the Membership shall be required for removal.
Upon removal, the officer shall become an Active Member in good standing
with CMAC and no other penalties within the group shall be enforced
for his offenses as an officer.

% ** Succession
\subsection{Succession}

% *** President
\subsubsection{President}

In cases of temporary absence not lasting for more than one academic
quarter, the Treasurer shall serve as Acting President, and the Executive
Board shall, within one month of the beginning of the absence, vote
to appoint an Acting President for the duration of the absence. Such
an appointment shall be effective unless objected to, either verbally
or in writing, by one third of the voting members of the group, at
which time a special election for the position of Acting President
shall be called by the Treasurer. If the Executive Board, or possibly
the aforementioned special election, selects a current member of the
Executive Board, the fifth vote on the board will be granted to a
member of the group as defined below in 8.7.3.

% *** All Other Officers
\subsubsection{All Other Officers}

In the case of impeachment or any other inability to serve for the remainder of the year among the remainder
of the Executive Board, the President shall immediately hold a special election for the general membership
to elect a replacement officer by a majority vote, following the rules and regulations set forth by Section
8.5. In cases of temporary absence not lasting for more than one academic quarter, the President shall call
a special meeting in which an acting officer shall be elected for the duration of the absence.

% *** Temporary Executive Voting Privileges
\subsubsection{Temporary Executive Voting Priviledges}

If the Executive Board should find itself with four or fewer Officers
with Executive Voting rights, the Membership may grant Executive Voting
privileges to one of the Members by majority vote. No other Executive
privileges or duties will be granted to this Member. The Member shall
hold temporary Executive Voting privileges until there exist a full
Board with Executive Voting privileges.

% * Executive Cabinet
\section{Executive Cabinet}

The Executive Board may appoint Executive Cabinet members at its discretion
to administer the operational duties of the Executive Board. The Cabinet
may include the positions of Stage Manager, Special Event Chair,
Social Chair, Philanthropy Chair, Music Chair, Newsletter Chair, Webmaster,
and Librarian. Additional Executive Cabinet positions may be appointed
by the Executive Officers as they deem fit. Cabinet members shall
have no vote on the Executive Board. \\
 All Executive Cabinet Appointees shall:
\begin{enumerate}
\item be students of the University of Chicago in good standing;
\item if appointed, serve for no more than two consecutive years in the
same position;
\item be in Chicago for at least two quarters of the academic year for which
they are appointed; and
\item inform the Executive Board of any conflicts and prolonged absences
during the academic year for which they are appointed.
\end{enumerate}

% ** The Duties of the Executive Cabinet
\subsection{The Duties of the Executive Cabinet}

All Cabinet members shall complete all tasks assigned to their position
by the Executive Board and enumerated herein. Cabinet members will
be summoned to attend two Executive Board meetings per quarter, unless
instructed otherwise by the Executive Board, and are encouraged to
attend all Executive Board meetings.

% ** Description of Positions
\subsection{Description of Positions}


\subsubsection{Stage Manager}

The Stage Manager shall be responsible for the logistics of the quarterly
concert. His specific duties shall include, but not be limited to,
coordinating ushers, compiling programs, and preparing performance
and rehearsal space. He will be required to attend all Executive Board
meetings as requested. He shall report to the President.


\subsubsection{Ticket Czar}

The Ticket Czar shall be responsible for printing, distributing, and
tracking tickets for quarterly performances. He shall also be responsible
for collecting ticket money from all the members. He shall report
to the Publicity Chair. He shall present a record of ticket sales
to the Treasurer.


\subsubsection{Social Chair}

The Social Chair shall oversee all issues relating to the social life
and atmosphere of CMAC. This includes, but is not limited to, the
maintenance and development of CMAC traditions, the planning and execution
of social events, and the regulation of the social atmosphere of CMAC.
He shall report to the Development Chair.


\subsubsection{Philanthropy Chair}

The Philanthropy Chair shall coordinate all philanthropic and charity
efforts of CMAC. This includes, but is not limited to, concerts for
charity, community service days, food and clothing drives, and charitable
donations. He shall report to the Development Chair.


\subsubsection{Music Chair}

The Music Chair shall collaborate closely with the Conductor in maintaining
and developing the musical integrity of CMAC. This includes, but is
not limited to, aiding the President in running rehearsal, conducting
the group in the absence of the Conductor and coordinating Sectionals
with Sectional Leaders. In all musical matters he shall defer to the
judgment of the Conductor. He shall report to the Conductor.


\subsubsection{Newsletter Chair}

The Newsletter Chair shall be responsible for the creation and publishing
of an annual CMAC newsletter. He shall report to the Development Chair.


\subsubsection{Webmaster}

The Webmaster shall be responsible for all oversight regarding CMAC's
domain, www.cmacsings.com, and other web presence. He shall report to the
Publicity Chair.


\subsubsection{Librarian}

The Librarian shall be responsible for maintaining the CMAC catalogue
of music. This includes, but is not limited to, copying and handing
out all new music, and maintaining the records of all old music. He
shall report to the Conductor.


\subsubsection{Special Event Chair}

The Special Event Chair shall be responsible for coordinating all
details related to a specific upcoming special event. The Executive
Board shall enumerate the specific duties of this position based on
the nature of the special event.


\subsubsection{Design Chair}

The Design Chair shall be responsible for designing advertising materials,
including but not limited to posters, invitations, and tickets. He
shall report to the Publicity Chair.

% ** Appointment of the Executive Cabinet
\subsection{Appointment of Executive Cabinet}

With the exception of the Special Events Chair, the Executive Board
shall appoint all Cabinet Members following the Annual Meeting of
CMAC. The Executive Board shall announce cabinet positions no later
than two weeks prior to when appointments are made. Members who are
interested in an Executive Cabinet position are encouraged to make
their desire known to the Executive Board a week before appointments
are made.

% *** Appointment of the Special Event Chair
\subsubsection{Appointment of the Special Event Chair}

The Executive Board shall appoint a Special Event Chair when appropriate.

% *** Appointment of Music Chair
\subsubsection{Appointment of Music Chair}

The Executive Board shall consult with the Conductor when appointing
the Music Chair.

% *** Subsumed Positions
\subsubsection{Subsumed Positions}

In the event of absence of qualified or interested members for a given
Cabinet Position, the duties of the position shall be subsumed by
the Executive Board.

% * Section Leaders
\section{Section Leaders}

% ** Election of Section Leaders
\subsection{Election of Section Leaders}

The Members shall hold elections for section leaders by the fifth
regularly scheduled rehearsal of the year. The Section Leaders shall
be elected, respectively, by the Members of each Section for a term
of one year. Section Leaders must have been Active Members for at
least one academic quarter. Officers are eligible for Section Leader,
but the Membership is encouraged to elect non-Officer Members.

% ** Emergency Election
\subsection{Emergency Election}

In the case of a vacancy occurring more than one month before the
expiration of the term of a Section Leader, the Members of that Section
shall fill such vacancy by special election held as soon as possible
after the vacancy occurs; any Section Leader so elected shall qualify
immediately upon election and shall hold office for the remainder
of the unexpired term.

% ** Duties of Section Leaders
\subsection{Duties of Section Leaders}

Section Leaders shall be the musical leaders of their respective Sections,
and shall:
\begin{enumerate}
\item be the musical and social leaders of their respective section;
\item be responsible for assisting the Conductor in achieving the musical
excellence of the group;
\item maintain clear and accurate records of music rehearsed and musical
notations given during Rehearsals;
\item communicate the same to absent Members of their Sections and otherwise
assist them musically;
\item assist the Secretary in the maintenance of attendance records, the
taking of active steps to discourage absences, and the dissemination
of information to the Members;
\item have the authority to call meetings of their respective sections for
extra practice and to chair or to appoint a chair for said meetings;
\item be responsible for the maintenance and development of traditions for
their section;
\item coordinate with the Social Chair to plan and execute social events
for their section;
\item coordinate with other Section Leaders and with the Social Chair to
plan and execute competitive activities between sections; and
\item he shall attend the first Executive Board meeting after his election,
as well as two more Executive Board meetings in the same quarter,
and two in each of the following quarters.
\end{enumerate}

% ** Presidential Chair of Section Meetings
\subsection{Presidential Chair of Section Meetings}

At his discretion, the President shall chair (or appoint a chairman
for) any meeting of Members of a Section that the Section Leader cannot
or should not chair.

% ** Succession of Section Leaders
\subsection{Succession of Section Leaders}

If a Section Leader is for any reason unable to serve, there shall
be an immediate vote held by the section to nominate a successor and
replace the Section Leader.

% * Conductor
\section{Conductor}

CMAC shall have a Conductor responsible for running all rehearsals
and performances. He shall be appointed by the Board of Directors
and shall have a contract created or renewed each year by them, regarding
expectations for his role and possible compensation for his activities.
He shall not be a member of the Executive Board and shall have no
vote, but he shall always have the right to attend their meetings
and express his opinions therein.

% * Finances
\section{Finances}

% ** Signator
\subsection{Signator}

The Treasurer shall be designated signator of the CMAC Checking Account
and all other Operating Accounts of the group. Any and all purchases
and/or expenditure from this account shall be made only with the consent
of the Board. The Chairman of the Board shall be the signator of the
Alumni Account along with the Treasurer.

% ** Assessment
\subsection{Assessment}

The Treasurer, with the consent of the Board, shall have the authority
to prescribe reasonable assessments of the Active Membership including
trip expenses and other necessary items. These assessments shall be
collected by the Treasurer or designee.

% ** Dues Request
\subsection{Dues Request}

The Treasurer shall submit to the Board an amount to request from
the Membership in member dues. Upon approval of the Board, the Treasurer
or designee shall collect these dues.

% ** Appeal of Assessments
\subsection{Appeal of Assessments}

Any Member may appeal any financial obligations to the Board within
two weeks of the assessment. The Board may decide to delay, reduce,
or cancel any financial obligation.

% ** Revenues
\subsection{Revenues}

The Ticket Czar shall oversee the distribution and sale of concert
tickets and shall transfer all monies raised to the Treasurer. The
Development Chair shall be responsible for the distribution and sale
of recordings or other merchandise. All sales shall be made at a price
determined by the Executive Board.

% ** Fundraising
\subsection{Fundraising}

The Development Chair shall have the authority to raise funds for
CMAC through any and all available means, upon such approval by the
Board of Directors. The Development Chair shall coordinate the contact
of Alumni in order to solicit contributions to the CMAC Operating
Accounts a minimum of once per academic year. He shall delegate these
responsibilities as needed.

% ** CMAC Alumni Accounts and Endowment
\subsection{CMAC Alumni Accounts and Endowment}

The Treasurer shall make regular reports on the status of CMAC's Capital
Campaigns and Endowment accounts to the Board, and shall be responsible
for making requests to CMAC's Alumni Accounts for expenditures from
them. Any and all requests for expenditures of these accounts by the
Treasurer must be approved by the Board.

% ** Financial Transparency
\subsection{Financial Transparency}

By the fifth rehearsal of each quarter, the treasurer shall be responsible
for presenting to the membership the status of CMAC's treasury, including
but not limited to: concert revenues, current balance, and projected
cashflow.

% * Indemnification
\section{Indemnification}

To the fullest extent now or hereafter permitted by the laws of Illinois,
CMAC shall indemnify any person who is or was an Officer against expenses
and liabilities in connection with any proceeding involving such Officer
by reason of his having been in such a position.

% * Ratification and Amendment
\section{Ratification and Amendment}
\begin{enumerate}
\item These Bylaws shall be ratified by a two-thirds majority of the members
and shall become effective immediately thereafter.
\item All amendments or additions to these Bylaws shall be ratified by a
two-thirds majority of the Members present, a quorum requiring three-fourths
of the Active Members.
\end{enumerate}

% * Discrimination
\section{Discrimination}

CMAC considers students on the basis of individual
merit and without regard to race, color, religion, sex, gender, sexual
orientation, national or ethnic origin, age, disability, or other
factors irrelevant to participation in CMAC. However,
CMAC may make requirements based on vocal range or quality that may
result in the group's being composed entirely of one
sex. The University of Chicago has other all-female a cappella groups
which provide a substantially equal single-sex environment for female
students.
\end{document}
